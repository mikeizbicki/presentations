\begin{frame}[fragile]{The simplified cover tree}

\begin{center}
\begin{tikzpicture}
    [ draw
    , every node/.style={minimum size=10mm,fill=white}
    , level/.style={sibling distance = 23mm/#1, level distance=12mm}
    %,    level distance = 1.5cm}
    , sibling distance=8mm
    ]
\draw (-2.3,0) -- (8.6,0)[dotted];
\draw (-2.3,-12mm) -- (8.6,-12mm)[dotted];
\draw (-2.3,-24mm) -- (8.6,-24mm)[dotted];
\node[shape=circle,draw] at (2.5,0) {10}
    child { node[circle,draw] {8}
        child { node[circle,draw] {7}  }
        child { node[circle,draw] {9} }
        }
    child { node[circle,draw] {12}
        %child { node[circle,draw,fill=lightred,line width=1pt] {9}  }
        %child { node[circle,draw] {13} }
        }
    ;
%\node[shape=circle,draw] at (5,0) {10}
    %child { node[circle,draw] {8}
        %child { node[circle,draw] {7}  }
        %child { node[circle,draw,fill=lightgreen,line width=1pt] {9} }
        %}
    %child { node[circle,draw] {12}
        %child { node[circle,draw,fill=lightgreen,line width=1pt] {11}  }
        %child { node[circle,draw] {13} }
        %}
    %;
\node[fill=none] at (8,3mm) {level 3};
\node[fill=none] at (8,-9mm) {level 2};
\node[fill=none] at (8,-21mm) {level 1};
\end{tikzpicture}
\end{center}

\uncover<1> {
%\vspace{0.15in}
\textbf{The covering invariant.}
For every node $p$, define the function $\covdist p = \exprad p^{\level p}$.
For each child $q$ of $p$
$$
\dist p q \le \covdist p
$$

\vspace{0.15in}
\textbf{The separating invariant.}
For every node $p$, define the function $\sepdist p = \exprad p^{\level p-1}$.
For all distinct children $q_1$ and $q_2$ of $p$
$$
\dist{q_1}{q_2} \ge \sepdist p
$$
}

\uncover<2> {
\vspace{-1.70in}
Advantages of the simplified cover tree:
\vspace{0.05in}
\begin{itemize}
\item Maintains all runtime guarantees of the original cover tree.
\vspace{0.05in}

\item Significantly easier to understand and implement.

%\vspace{0.05in}
The original cover tree was described in terms of an infinitely large tree, only a subset of which actually gets implemented.

\vspace{0.05in}
\item Requires exactly $n$ nodes instead of $O(n)$ nodes.

%\vspace{0.05in}
Fewer nodes means a faster constant factor for all algorithms. %insertions and queries are faster.
\end{itemize}
\vspace{1.3in}
}

\end{frame}


\begin{frame}[fragile]{The simplified cover tree}

\centering
\graphicspath{{slides/paperimg/}}
\input{slides/paperimg/nodes}

\end{frame}
