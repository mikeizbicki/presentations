\begin{frame}[fragile]{Cache oblivious cover tree}

%RAM model is a lie
%
%\vspace{0.1in}
Need to consider cache accesses for fast, modern data structures

%\vspace{0.1in}
%In a \emph{cache oblivious} data structure, the programmer doesn't need to know any details about the underlying hardware
%
\begin{center}
\includegraphics[width=10cm]{slides/cpu_cache_structure}
\end{center}

{\tiny image from: \url{http://1024cores.net} }
\end{frame}

%%%%%%%%%%%%%%%%%%%%%%%%%%%%%%%%%%%%%%%%%%%%%%%%%%%%%%%%%%%%%%%%%%%%%%%%%%%%%%%%

\begin{frame}[fragile]{Cache oblivious cover tree}

%The trick: use the van Emde Boas tree layout
Arrange nodes in memory according to a preorder traversal of the tree

(van Emde Boas \emph{et al.}, 1966; Demaine, 2002)
\vspace{0.1in}

\begin{center}
\includegraphics[width=6cm]{slides/preorder.png}
\end{center}

{\tiny image from: Wikipedia}
%(nodes arranged in memory according to a pre-order depth first traversal)
%
%\vspace{-0.1in}
%%{
%%\centering
%\begin{center}
%\includegraphics[width=8cm]{slides/cache_oblivious_divide}
%\end{center}
%%}
%
%\vspace{-0.1in}
%Optimal cache performance without any knowledge of the architecture
%\vspace{0.15in}

%source: 1024cores.net

\end{frame}

%%%%%%%%%%%%%%%%%%%%%%%%%%%%%%%%%%%%%%%%%%%%%%%%%%%%%%%%%%%%%%%%%%%%%%%%%%%%%%%%

\begin{frame}[fragile]{The cache efficiency of three cover tree implementations}

\centering
\graphicspath{{slides/paperimg/}}
\input{slides/paperimg/cache-hlearn}
\definecolor{colorOrig}{RGB}{102,51,0}
\definecolor{colorMlpack}{RGB}{204,153,0}
\begin{tikzpicture}
    %\node[draw,fill=colorOrig,minimum width=0.05in,minimum height=0.3in] at (0,0) {};
    %\node at (0,0.79in) {\small\rotatebox{90}{Reference cover tree}};
    %\node[draw,fill=colorMlpack,minimum width=0.05in,minimum height=0.3in] at (0.15in,0) {};
    %\node at (0.15in,0.79in) {\small\rotatebox{90}{MLPack's cover tree}};
    %\node[draw,fill=blue,minimum width=0.05in,minimum height=0.3in] at (0.3in,0) {};
    %\node at (0.3in,0.98in) {\small\rotatebox{90}{Our cover tree (unpacked)}};
    %\node[draw,fill=lightblue,minimum width=0.05in,minimum height=0.3in] at (0.45in,0) {};
    %\node at (0.45in,0.91in) {\small\rotatebox{90}{Our cover tree (packed)}};
    \node[draw,fill=blue,minimum width=0.05in,minimum height=0.3in] at (0.3in,0) {};
    \node at (0.3in,0.98in) {\small\rotatebox{90}{Without van embde boas }};
    \node[draw,fill=lightblue,minimum width=0.05in,minimum height=0.3in] at (0.45in,0) {};
    \node at (0.45in,0.88in) {\small\rotatebox{90}{With van embde boas }};
    \node at (0,-0.475in) {};
\end{tikzpicture}

\vspace{0.15in}
Measured using Linux's \lstinline{perf stat} utility on an Amazon AWS instance
\end{frame}

%%%%%%%%%%%%%%%%%%%%%%%%%%%%%%%%%%%%%%%%%%%%%%%%%%%%%%%%%%%%%%%%%%%%%%%%%%%%%%%%
%
%\begin{frame}[fragile]{The cache efficiency of three cover tree implementations}
%
%\centering
%\graphicspath{{slides/paperimg/}}
%\input{slides/paperimg/stalled-front}
%\definecolor{colorOrig}{RGB}{102,51,0}
%\definecolor{colorMlpack}{RGB}{204,153,0}
%\begin{tikzpicture}
    %\node[draw,fill=colorOrig,minimum width=0.05in,minimum height=0.3in] at (0,0) {};
    %\node at (0,0.79in) {\small\rotatebox{90}{Reference cover tree}};
    %\node[draw,fill=colorMlpack,minimum width=0.05in,minimum height=0.3in] at (0.15in,0) {};
    %\node at (0.15in,0.79in) {\small\rotatebox{90}{MLPack's cover tree}};
    %\node[draw,fill=blue,minimum width=0.05in,minimum height=0.3in] at (0.3in,0) {};
    %\node at (0.3in,0.98in) {\small\rotatebox{90}{Our cover tree (unpacked)}};
    %\node[draw,fill=lightblue,minimum width=0.05in,minimum height=0.3in] at (0.45in,0) {};
    %\node at (0.45in,0.91in) {\small\rotatebox{90}{Our cover tree (packed)}};
    %\node at (0,-0.475in) {};
%\end{tikzpicture}
%
%\vspace{0.15in}
%Measured using Linux's \lstinline{perf stat} utility on an Amazon AWS instance
%\end{frame}
